\documentclass{beamer}
\usepackage[T1]{fontenc}
\usepackage[utf8]{inputenc}
\usepackage{lmodern}
\usepackage{tikz}
\usepackage{subfigure}
\usepackage[italian]{babel}
\usepackage{listings}

\graphicspath{{Images/}}

\mode<presentation>
{   
   \setbeamercovered{transparent}
   \usefonttheme{serif}

%---
%\usetheme{Berlin}
%\usetheme{PaloAlto}
%\usetheme{Malmoe}
%\usetheme{Warsaw}
%---
%\usecolortheme{wolverine}
%\setbeamertemplate{blocks}[rounded][shadow=true]
%\usecolortheme{sidebartab}
%\setbeamercolor{some beamer element}{fg=red}
}
% elimina controlli in basso a destra
%\beamertemplatenavigationsymbolsempty 

%------------------------------------------------------------------------
%  Riduce dimensioni delle didascalie
%------------------------------------------------------------------------
\setbeamerfont{caption}{size=\tiny}
%------------------------------------------------------------------------
%  Lista con segni delle carte
%------------------------------------------------------------------------
\newcommand*\lista{\item[$\diamondsuit$]}
%------------------------------------------------------------------------
%  Fa ricomparire l'indice a ogni cambio sezione
%------------------------------------------------------------------------
\AtBeginSection[] {
	\begin{frame}<beamer>{Outline}
		\tableofcontents[currentsection,currentsubsection]
	\end{frame}
}
%------------------------------------------------------------------------
%  Listing style
%------------------------------------------------------------------------
\lstset{ language=Lisp, stringstyle=\ttfamily, basicstyle=\scriptsize,showstringspaces=false,
         otherkeywords={slot,defrule,deftemplate,else,if,bind,retract},breaklines=true }
%------------------------------------------------------------------------
\title{Briscola in 5: un sistema esperto per una strategia a regole}
\author{Mattia Vinci}
\date{Venerdì 17 luglio 2015}
\titlegraphic{\includegraphics[width=2cm]{logo}}
\institute[Università degli studi di Torino]
{\textsc{Università degli studi di Torino} \\ Dipartimento di Informatica \\ \medskip Relatore: Dott. Roberto Micalizio}
%------------------------------------------------------------------------



\begin{document}


%------------------------------------------------------------------------
%  FRONTESPIZIO
%------------------------------------------------------------------------
\begin{frame}
   \titlepage
\end{frame}

%------------------------------------------------------------------------
%  SOMMARIO
%------------------------------------------------------------------------
\section*{Sommario}
\begin{frame}
   \frametitle{Sommario}
   \begin{center}
   \textbf{Scopo del lavoro:}   
   \end{center}
   Costruzione di una piattaforma per il gioco \emph{Briscola in 5} per giocatori umani e virtuali.
   \vspace*{1em}
   \begin{center}
      \textbf{Contributi realizzati:}
   \end{center}
   \begin{itemize}
      \lista Intelligenza artificiale per il giocatore virtuale
      \lista Piattaforma per il gioco
   \end{itemize}  
   
\end{frame}

%------------------------------------------------------------------------
%  INDICE
%------------------------------------------------------------------------
\section*{Indice}
\begin{frame}
   \frametitle{Indice}
   \tableofcontents
\end{frame}

%------------------------------------------------------------------------
%  LA BRISCOLA IN 5
%------------------------------------------------------------------------
\section{La briscola in 5}

\begin{frame}
   \frametitle{Il gioco}
   
   La \emph{briscola in 5} è un gioco di carte derivato dalla briscola classica, ma con qualche significativa differenza:\\
   \begin{itemize}
      \lista le squadre non sono simmetriche
      \lista vi è una fase d'asta iniziale
      \lista la formazione delle squadre non è nota a tutti
   \end{itemize}
\end{frame}

%------------------------------------------------------------------------
%  L'INTELLIGENZA ARTIFICIALE
%------------------------------------------------------------------------
\section{L'intelligenza artificiale}

\begin{frame}
   \frametitle{Background}
   Abbiamo inizialmente passato in rassegna alcuni fra i principali metodi presenti in letteratura per la risoluzione di giochi.\\
   \begin{itemize}
      \lista Ricerca nello spazio degli stati e algoritmo \texttt{minimax} 
      \lista Teoria dei giochi
      \lista Sistemi esperti
   \end{itemize}
   
\end{frame}

%------------------------------------------------------------------------

\begin{frame}
   \frametitle{Il problema della briscola in 5}
   Non essendo le squadre note a tutti i giocatori, alcuni di essi si trovano a giocare con persone (o agenti virtuali) che non possono identificare con certezza nè come compagni nè come avversari.
   \vfill
   \pause
   Per questo motivo l'ambiente della briscola in 5 non può essere considerato nè come \emph{competitivo}, nè come \emph{cooperativo}, rendendo inapplicabili i metodi di ricerca nello spazio degli stati e la teoria dei giochi a causa dell'impossibilità di assegnare \emph{valori di utilità} ben definiti alle mosse.
   \vfill
   \pause   
   Questo ci ha spinti alla decisione di tentare di simulare il comportamento di un giocatore umano tramite un \emph{sistema esperto}.
   
\end{frame}

%------------------------------------------------------------------------

\begin{frame}
   \frametitle{Il sistema esperto}
   Il sistema esperto è stato realizzato tramite il \emph{rule engine} \textsc{Jess}.
   \vfill
   \pause
   Il framework generale realizzato prevede due tipi di strategie:
   \begin{itemize}
      \lista di \textbf{analisi}: assegnano empiricamente la ``probabilità'' che un giocatore abbia un certo ruolo
      \lista di \textbf{decisione}: selezionano la carta da giocare
   \end{itemize}
   \vfill
   \pause
   Più regole di decisione possono essere attivate contemporaneamente; un sistema di \emph{conflict resolution} basato sulla priorità assegnata alle regole decide quale carta tra le candidate.
\end{frame}

%------------------------------------------------------------------------
%  LA PIATTAFORMA
%------------------------------------------------------------------------
\section{La piattaforma}

\begin{frame}
   \frametitle{Architettura della piattaforma}
   Requisiti:
   \begin{itemize}
      \lista distribuita
      \lista interfaccia utente esperto
      \lista log attività
   \end{itemize}
   \vfill   
   \pause
   La piattaforma è stata realizzata usando il framework per lo sviluppo di piattaforme multi-agente \textsc{Jade}.\\
   Prevede due tipi di agente: \emph{mazziere} e \emph{giocatore}.
   \visible<2>{
   \begin{center}
   \begin{figure}      
   \includegraphics[width=.8\textwidth]{architettura}
   \end{figure}
   \end{center}
   }
\end{frame}

%------------------------------------------------------------------------

\begin{frame}
   \frametitle{L'agente Mazziere}
   L'agente mazziere regola e gestisce la partita. Fra i suoi compiti principali:
   \begin{itemize}
      \pause
      \lista ``aprire un tavolo'' (servizio pagine gialle di \textsc{Jade})
      \pause
      \lista farsi carico della comunicazione fra agenti (tranne chat)
      \pause
      \lista gestire la partita tramite messaggi
      \pause
      \lista redigere un file di log
   \end{itemize}
   \vfill
   \pause
   La presenza dell'agente mazziere rende centralizzata la gestione della partita. 
   Questo presenta alcuni vantaggi:
   \begin{itemize}
      \pause
      \lista rispetto delle regole
      \pause
      \lista semplice gestione di piattaforme con molti ``tavoli aperti'' e tornei
      \pause
      \lista unico file di log da parte di un giocatore onniscente
   \end{itemize}
\end{frame}

%------------------------------------------------------------------------

\begin{frame}
   \frametitle{Giocatore}
   L'agente giocatore è quello che prende parte al gioco e s'impegna a fare una mossa legale quando gli è richiesta.\\
   \vfill
   \pause   
   Può avere modalità:
   \begin{itemize}
      \lista manuale
      \lista random
      \lista strategia da file
   \end{itemize}
   \vfill
   \pause
   Nella fase di sperimentazione abbiamo usato un file contenente una trentina di regole generali raccolte su manuali, siti e forum dedicati alla briscola in 5, oltre che dall'esperienza personale.
\end{frame}

%------------------------------------------------------------------------
%  CONCLUSIONI
%------------------------------------------------------------------------

\section{Risultati sperimentali e conclusioni}
\begin{frame}
   \frametitle{Risultati}
   Abbiamo condotto degli esperimenti facendo giocare delle partite a dei giocatori con una strategia basilare contro dei giocatori random in diverse configurazioni.
   Alcuni fra i risultati più significativi:
   
   
\def\angle{0}
\def\radius{3}
\def\cyclelist{{"blue","red","green"}}
\newcount\cyclecount \cyclecount=-1
\newcount\ind \ind=-1


\begin{columns}
   \column{.5\textwidth}
   \vfill
   
\begin{figure}
   \tiny
   \centering
\begin{tikzpicture}[scale=.37]
  \foreach \percent/\name in {
      66.7/Soci,
      30/Villani,
      3.3/Pareggio
    } {
      \ifx\percent\empty\else               % If \percent is empty, do nothing
        \global\advance\cyclecount by 1     % Advance cyclecount
        \global\advance\ind by 1            % Advance list index
        \ifnum3<\cyclecount                 % If cyclecount is larger than list
          \global\cyclecount=0              %   reset cyclecount and
          \global\ind=0                     %   reset list index
        \fi
        \pgfmathparse{\cyclelist[\the\ind]} % Get color from cycle list
        \edef\color{\pgfmathresult}         %   and store as \color
        % Draw angle and set labels
        \draw[fill={\color!50},draw={\color}] (0,0) -- (\angle:\radius)
          arc (\angle:\angle+\percent*3.6:\radius) -- cycle;
        \node at (\angle+0.5*\percent*3.6:0.7*\radius) {\percent\,\%};
        \node[pin=\angle+0.5*\percent*3.6:\name]
          at (\angle+0.5*\percent*3.6:\radius) {};
        \pgfmathparse{\angle+\percent*3.6}  % Advance angle
        \xdef\angle{\pgfmathresult}         %   and store in \angle
      \fi
    };
\end{tikzpicture}
\vfill
\vspace*{2.2em}
\caption{Random vs Random}
\end{figure}
%---
\column{.5\textwidth}

\begin{figure}

\subfigure{
\tiny
\cyclecount=-1
\ind=-1
\begin{tikzpicture}[scale=.3]%
  \foreach \percent/\name in {%
      80/Vittorie,%
      20/Sconfitte,%
      0/Pareggi%
    } {%
      \ifx\percent\empty\else%               % If \percent is empty, do nothing
        \global\advance\cyclecount by 1%     % Advance cyclecount
        \global\advance\ind by 1%            % Advance list index
        \ifnum3<\cyclecount%                 % If cyclecount is larger than list
          \global\cyclecount=0%              %   reset cyclecount and
          \global\ind=0%                     %   reset list index
        \fi%
        \pgfmathparse{\cyclelist[\the\ind]}% Get color from cycle list
        \edef\color{\pgfmathresult}%         %   and store as \color
        % Draw angle and set labels
        \draw[fill={\color!50},draw={\color}] (0,0) -- (\angle:\radius)%
          arc (\angle:\angle+\percent*3.6:\radius) -- cycle;%
        \node at (\angle+0.5*\percent*3.6:0.7*\radius) {\percent\,\%};%
        %\node[pin=\angle+0.5*\percent*.5:\name]%
          at (\angle+0.5*\percent*3.6:\radius) {};%
        \pgfmathparse{\angle+\percent*3.6}%  % Advance angle
        \xdef\angle{\pgfmathresult}%         %   and store in \angle
      \fi%
    };%
\end{tikzpicture}%
(a)
}


\subfigure{%
\tiny
\cyclecount=-1
\ind=-1
\begin{tikzpicture}[scale=.3]%
  \foreach \percent/\name in {%
      60/Vittorie,%
      40/Sconfitte,%
      0/Pareggi%
    } {%
      \ifx\percent\empty\else%               % If \percent is empty, do nothing
        \global\advance\cyclecount by 1%     % Advance cyclecount
        \global\advance\ind by 1%            % Advance list index
        \ifnum3<\cyclecount%                 % If cyclecount is larger than list
          \global\cyclecount=0%              %   reset cyclecount and
          \global\ind=0%                     %   reset list index
        \fi%
        \pgfmathparse{\cyclelist[\the\ind]}% Get color from cycle list
        \edef\color{\pgfmathresult}%         %   and store as \color
        % Draw angle and set labels
        \draw[fill={\color!50},draw={\color}] (0,0) -- (\angle:\radius)%
          arc (\angle:\angle+\percent*3.6:\radius) -- cycle;%
        \node at (\angle+0.5*\percent*3.6:0.7*\radius) {\percent\,\%};%
        %\node[pin=\angle+0.5*\percent*.5:\name]%
          at (\angle+0.5*\percent*3.6:\radius) {};%
        \pgfmathparse{\angle+\percent*3.6}%  % Advance angle
        \xdef\angle{\pgfmathresult}%         %   and store in \angle
      \fi%
    };%
\end{tikzpicture}%
(b)
}%
\caption{Una squadra a strategie contro una random. In (a) i soci a strategie; in (b) villani a strategie. In Blu le vittorie, in rosso le sconfitte.}
\end{figure}
   
   
   
\end{columns}



   
   
\end{frame}

%------------------------------------------------------------------------

\begin{frame}
   \frametitle{Conclusioni}
   Abbiamo realizzato una piattaforma multiagente distribuita per l'esecuzione di partite di briscola in 5 cui possano partecipare giocatori sia umani che virtuali.\\
   Viste le peculiarità della briscola in 5, per l'intelligenza artificiale del giocatore virtuale abbiamo optato per un sistema esperto.\\
   Abbiamo scritto un framework in \textsc{Jess} che permetta l'implementazione di strategie in maniera semplice, insieme ad un set di strategie di carattere generale.\\
   Abbiamo implementato la piattaforma --- usando \textsc{Jade} --- con due tipi di agente, il \emph{mazziere} e il \emph{giocatore} e prevedendo un'interfaccia per l'utente esperto che permetta l'annotazione e il suggerimento di nuove regole.\\
   Infine abbiamo effettuato delle sperimentazioni i cui risultati incoraggiano ulteriori sviluppi.   
\end{frame}

%------------------------------------------------------------------------

\begin{frame}
   \frametitle{Possibili sviluppi futuri}
   \begin{itemize}
      \lista Incorporazione del sistema esperto con un altro metodo nella seconda fase della partita.
      \lista Utilizzo di metodi di \emph{machine learning} per la raccolta di strategie.
      \lista Integrazione di un meccanismo di \emph{reputation} alle strategie di analisi.
      \lista Nuovi criteri di valutazione della validità delle strategie, compreso il confronto con giocatori umani.
   \end{itemize}
   
\end{frame}

%------------------------------------------------------------------------

\begin{frame}
   \centering
   Grazie per l'attenzione.
\end{frame}



\begin{frame}[fragile]
   \frametitle{Il sistema esperto: rappresentazione della conoscenza}
\begin{lstlisting}
(deftemplate in-mano "carte che posso ancora giocare"
   (slot card)    (slot rank)    (slot suit) )

( deftemplate giocata "info sulle giocate"
    (slot player)	(slot card)		(slot rank)    (slot suit)
    (slot mano)   (slot turno)	(slot tipo) )

( deftemplate posso-prendere "Carte con le quali posso prendere la mano"
    (slot card)   (slot rank)    (slot suit)   )

( deftemplate carichi-in-mano "Le carte da punto che ho in mano"
    (slot card)    (slot rank)    (slot suit)    (slot points) )

( deftemplate giaguaro  (slot player) )
( deftemplate socio (slot player) )
( deftemplate villano   (slot player) )
( deftemplate seme-mano-fact (slot suit) )
( deftemplate prob-socio (slot player) (slot sal) )
\end{lstlisting} 
\end{frame}


\begin{frame}[fragile]
   \frametitle{Il sistema esperto: una regola di decisione}
   Se sono il socio e gioco subito dopo al chiamante, prendo per lasciarlo ultimo la mano successiva, soprattutto nelle mani finali.
\begin{lstlisting}
( defrule socio-tiene-giaguaro-ultimo
    ?w <- (calcola-giocata)
    (mio-ruolo socio)
    (giaguaro (player ?g))
    (mio-turno-numero ?n)
    (turno (player ?player&:(= ?player ?g)) (posizione ?pos&:(= ?n (mod (+ ?pos 1) 5) ) ) )
    (briscola (card ?b))
    (posso-prendere (card ?c&:(<> ?c ?b)))
=>
    (gioca ?c (- 100 (?c getValue)))
    (assert (ora-di-giocare))
)
\end{lstlisting}
\end{frame}



\begin{frame}[fragile]
   \frametitle{Il sistema esperto: una regola di decisione}
   Se un giocatore prende lasciando ultimo il giaguaro la mano successiva, probabilmente è il socio, soprattutto se nelle ultime mani.

\begin{lstlisting}
( defrule vs-socio-tiene-giaguaro-ultimo
    (not(exists(socio (player ?player))))
    (mano-numero ?mano-numero)
    (giaguaro (player ?g))
    (turno (player ?p&:(= ?p ?g)) (posizione ?pos-giaguaro))
    (giocata (player ?soc) (tipo ?t&:( or ( = ?t "taglio") (or (= ?t "strozzino")  (= ?t "strozzo") ) )  ))
    (turno (player ?pl&:(= ?pl ?soc)) (posizione ?pos-soc&:( = ?pos-soc (mod (+ ?pos-giaguaro 1) 5) )))
=>
    (if (> ?mano-numero 3) then
        (bind ?new-sal 80)
    else
        (bind ?new-sal 40)
    )
    (aumenta-sal-socio ?soc ?new-sal)
)
\end{lstlisting}
\end{frame}


\end{document}
