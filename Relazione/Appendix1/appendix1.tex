\chapter*{Appendice A: \textsc{Jade}}
\stepcounter{chapter}
\addcontentsline{toc}{chapter}{Appendice A: Jade}

\graphicspath{{Appendix1/figures/}}

\textsc{Jade}, acronimo di \emph{Java Agent DEvelopment Framework}, è un framework sviluppato interamente in \textsc{Java} atto a essere utilizzato come \emph{middle-ware} nello sviluppo di applicazioni multiagente.\\
È implementato rispettando le specifiche \emph{FIPA}, acronimo di \emph{Foundation for Intelligent Physical Agents}, un'organizzazione internazionale per lo sviluppo di standard per l'implementazione e la comunicazione di sistemi multi-agente.\\
\textsc{Jade} è free software, distribuito sotto la licenza LGPL e sviluppato da \textsc{Telecom Italia}.
Fornisce un sistema di astrazione per gli agenti,  un modello per la composizione ed esecuzione di \emph{task} e un servizio di \emph{pagine gialle}.
Le applicazioni multiagente sviluppate con \textsc{Jade} possono essere distribuite fra diversi host connessi in rete.
Su ogni macchina che esegue la piattaforma viene attivata una sola applicazione \textsc{Java} all'interno della JVM che funge da ``contenitore'' di agenti fornendo anche un ambiente completo per l'esecuzione concorrente di più agenti contemporaneamente.
% \begin{figure}[!htbp]
%   \begin{center}
%     \leavevmode
%       \includegraphics[width=0.8\textwidth]{distributed}
%     \caption{Architettura distribuita di \textsc{Jade}}
%     \label{fig:distributed}
%   \end{center}
% \end{figure}
% ------------------------------------------------------------------------
\\L'architettura di comunicazione offre un sistema di messaggistica flessibile ed efficiente: \textsc{Jade} crea e gestisce una coda privata di messaggi ACL per ogni agente; gli agenti possono accedere alla propria coda in diversi modi.\\
Gli agenti sono implementati come un singolo thread per agente, ma con la possibilità di associare ad ognuno più \emph{behaviour}, ciascuno dei quali viene eseguito in un thread dedicato, grazie all'ambiente multi-thread offerto da \textsc{Java}.\\
La piattaforma fornisce anche un'interfaccia grafica (GUI) per la gestione remota degli agenti, il loro controllo e monitoraggio.
