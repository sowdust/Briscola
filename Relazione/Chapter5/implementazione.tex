\chapter*{Implementazione degli agenti}
\stepcounter{chapter}
\addcontentsline{toc}{chapter}{Implementazione degli agenti}
\graphicspath{{Chapter5/Chapter5Figs/PNG/}{Chapter5/Chapter5Figs/PDF/}{Chapter5/Chapter5Figs/}}



Per l'implementazione degli agenti si è scelto di basarsi direttamente sulla piattaforma \textsc{Jade} sottostante, che fornisce una classe \texttt{Agent}.
Per questo si è creato la classe \texttt{GeneralAgent}, che estende appunto \texttt{Agent} di \textsc{Jade}.\\
Questa classe è a sua volta estesa dalle classi \texttt{PlayerAgent} e \texttt{MazziereAgent} che implementano rispettivamente l'agente giocatore e l'agente mazziere.\\
In relazione a queste tre classi ne esistono altre tre che ne gestiscono l'interfaccia grafica.
Sono state inoltre ovviamente messe a disposizione degli agenti varie classi che rappresentino il mondo circostante: dalle carte, al tavolo, ai diversi giocatori.\\
Infine, gli agenti dispongono di classi utilizzate per mantenere la memoria della partita, in modo da poter usare queste informazioni per la fase di \emph{reasoning} nel caso degli agenti giocatori, oppure per redigere il file di log nel caso dell'agente mazziere.
