\section{L'agente PlayerAgent}

Anche la classe \texttt{PlayerAgent} è, come si è anticipato, un'estensione della più generale \texttt{GeneralAgent}; essa implementa l'agente giocatore, ridefinendo i principali metodi della classe \texttt{Agent} fornita da \textsc{Jade} e fornendone ulteriori di varia utilità.

\subsection{Avvio dell'agente: il metodo \texttt{setup}}

All'avvio l'agente giocatore riceve da linea di comando i seguenti parametri:
\begin{itemize}
   \item un \emph{nome} che verrà utilizzato all'interno della piattaforma di gioco
   \item un'indicazione sulla \emph{strategia} da seguire: questa può essere \emph{manuale} nel caso di un giocatore umano, \emph{random} per un gioco pseudo-casuale o il percorso ad un \emph{file} che contenga le regole per il sistema esperto
   \item un flag \emph{visible} che indichi se avviare o meno l'interfaccia grafica
\end{itemize}

Una volta impostati i succitati parametri, l'agente giocatore aggiunge alla propria lista di \emph{behaviours} quello per l'iscrizione alla piattaforma e la ricerca di un tavolo, che dà inizio al flusso sequenziale dei \emph{task} che regolano lo svolgimento della partita.


\subsection{Alcuni metodi che s'interfacciano con il reasoner}

Come precedentemente detto, l'agente giocatore si interfaccia direttamente con il sistema esperto attraverso i propri metodi; questo è possibile grazie alle \emph{API \textsc{Java}} fornite da \textsc{Jess}, che permettono di asserire e modificare fatti nella \emph{KB} così come eseguire funzioni.

Nel listato \ref{initMano} è possibile vedere come questo avvenga all'inizio di ogni mano: una volta che l'agente mazziere ha comunicato a tutti i giocatori il resoconto della mano precedente e l'imminente inizio di una nuova, questi eseguono il proprio metodo \texttt{initMano} che calcola il nuovo ordine da assegnare ai giocatori e poi asserisce le relative informazioni come fatti nella \emph{KB}.


\begin{lstlisting}[caption={il metodo \texttt{initMano} di PlayerAgent},label=initMano]
    public void initMano(int mano, Player next) throws JessException {

        //  comunichiamo al reasoner l'inizializzazione di una nuova mano
        Value v = new Funcall("init-mano", rete).arg(
            new Value(mano, RU.INTEGER)).execute(rete.getGlobalContext());

        //  assegnamento dell'ordine dei turni ai giocatori
        int firstIndex = -1;

        for (int i = 0; i < players.size(); ++i) {
            if (players.get(i).equals(next)) {
                firstIndex = i;
                break;
            }
        }

        for (int i = 0; i < players.size(); ++i) {
            Fact f = new Fact("turno", rete);
            Player p = players.get((i + firstIndex) % 5);
            f.setSlotValue("player", new Value(p));
            f.setSlotValue("posizione", new Value(i, RU.INTEGER));
            rete.assertFact(f);
            if (p.equals(this.getPlayer())) {
                rete.assertString("(mio-turno-numero " + i + ")");
            }
        }

        rete.run();
        gui().initMano(mano, next);
    }
\end{lstlisting}



La stessa cosa succede in risposta alla ricezione dell'informazione (da parte del mazziere) di una nuova giocata: in questo caso l'agente giocatore dovrà innanzitutto aggiornare le immediate informazioni che descrivono, all'interno della \emph{KB}, le carte in tavolo e quelle ancora coperte; inoltre, se la carta giocata è la \emph{chiamata}, vengono aggiornate anche le informazioni riguardanti i ruoli dei giocatori.\\
Inoltre viene aggiunto alla \emph{KB} anche un fatto complesso che riassuma la giocata, contenente le informazioni sul giocatore autore della stessa e la carta che è stata giocata; queste informazioni verranno poi integrate all'interno del reasoner dopo che questo avrà attivato le proprie regole di analisi e quindi classificato la giocata come appartenente a uno dei predefiniti gruppi (\emph{taglio}, \emph{carico}, \emph{liscio}...) (Parte del codice della funzione si trova in \ref{addGiocata})

%
% \begin{lstlisting}[caption={il metodo \texttt{initMano} di PlayerAgent},label=initMano]
%     /**
%      * Every time a card is played, we need to update our internal
%      * representation.
%      * - add the card onto the table
%      * - if it was us playing, remove the card from our hand
%      * - if it was others playing, remove the card from the hidden deck
%      * - if the cards was the briscola, we now know the teams
%      *
%      * @param counter
%      * @param justPlayer
%      * @param justCard
%      * @throws JessException
%      */
%     public void addGiocata(int mano, int counter, Player justPlayer,
%                            Card justCard) throws
%         JessException {
%         //  aggiunta nella KB del fatto complesso che descrive la giocata
%         Fact z = new Fact("nuova-giocata", rete);
%         z.setSlotValue("player", new Value(justPlayer));
%         z.setSlotValue("card", new Value(justCard));
%         z.setSlotValue("rank", new Value(justCard.getRank()));
%         z.setSlotValue("suit", new Value(justCard.getSuit()));
%         rete.assertFact(z);
%         
%         //  aggiornamento dei ruooli nel caso in cui sia stata giocata la briscola
%         if (justCard.equals(briscolaCard)) {
%             Fact s = new Fact("socio", rete);
%             s.setSlotValue("player", new Value(justPlayer));
%             rete.assertFact(s);
%             for (Player p : players) {
%                 if (!p.equals(justPlayer) && !p.equals(this.getPlayer())) {
%                     s = new Fact("villano", rete);
%                     s.setSlotValue("player", new Value(p));
%                     rete.assertFact(s);
%                 }
%             }
%         }
%         
%         /* ... */
%
%         rete.run();
%         ((PlayerGUI) gui).addGiocata(mano, counter, justPlayer, justCard);
%     }
% \end{lstlisting}
%



Infine, anche l'oggetto \texttt{PlayerAgent} fa largo uso di classi di supporto utili soprattutto a memorizzare gli eventi della partita, sia per quanto riguarda la fase di gioco che per quella di asta.
