\section{MazziereAgent}

La classe \texttt{MazziereAgent} implementa evidentemente l'agente mazziere; come già detto, fa questo estendendo la classe \texttt{GeneralAgent}, dalla quale eredita i metodi base di un agente.\\

\subsection{Avvio dell'agente: il metodo \texttt{setup}}
La classe \texttt{Agent} di \textsc{Jade} fornisce un metodo \texttt{setup()} che viene chiamato quando la piattaforma avvia un nuovo agente.
\texttt{MazziereAgent} estende questo metodo per adattarlo alle proprie esigenze.\\
In questa fase (\ref{setup1}) l'agente mazziere analizza innanzitutto gli argomenti con cui è stato avviato:
\begin{enumerate}
   \item nome dell'agente
   \item scelta fra modalità grafica e testuale
   \item indirizzo del file di log
\end{enumerate}
inizializza le proprie strutture dati e il proprio reasoner (usato per codificare le regole del gioco).
\begin{lstlisting}[caption={stralcio del metodo setup() di MazziereAgent: inizializzazione strutture e reasoner},label=setup1]
        private static final String rulesFile = "briscola/reasoner/mazziere.clp";
        
        ...
        
        //  SETTING UP THE RETE INSTANCE FOR JESS RULE PROCESSING
        rete = new Rete();
        try {
            rete.batch(rulesFile);
            rete.reset();
        } catch (JessException ex) {
            System.out.println("Impossibile aprire il file " + rulesFile);
            ex.printStackTrace();
            takeDown();
        }
\end{lstlisting}


L'agente mazziere si registra al registro pubblico fornito da \textsc{Jade} per segnalare agli altri agenti della piattaforma la propria disponibilità a gestire una partita (\ref{setup2}). 

\begin{lstlisting}[caption={stralcio del metodo setup() di MazziereAgent: registrazione al servizio pagine gialle},label=setup2]
        private DFAgentDescription dfd;
        private ServiceDescription sd;
        
        ...
        
        //  REGISTER TO YELLOW PAGES
        dfd = new DFAgentDescription();
        dfd.setName(getAID());
        sd = new ServiceDescription();
        sd.setType(MAZZIERE);
        sd.setName(name);
        dfd.addServices(sd);
        try {
            DFService.register(this, dfd);
        } catch (FIPAException fe) {
            say("Errore durante la registrazione alle pagine gialle");
            fe.printStackTrace();
            takeDown();
        }
\end{lstlisting}

Infine, richiama il proprio metodo \texttt{startNewGame()} per dare inizio alla partita, avviando i \emph{Behaviours} che si occupano dell'apertura del tavolo e della ricezione degli eventuali commenti alle giocate.
Inizializza inoltre un nuovo "registro della partita" che costituisce un vero e proprio log delle attività svoltesi, che è implementato nella classe \texttt{GameMemory}.







\subsection{Chiusura dell'agente: il metodo \texttt{takeDown}}
Un altro metodo della classe \texttt{Agent} di \textsc{Jade} che viene qui esteso è il metodo chiamato alla chiusura dell'agente, ovvero \texttt{takeDown()}.
In questa fase (\ref{takedown}) l'agente mazziere si preoccupa di disallocare le proprie strutture e rimuovere il proprio nome dal registro pubblico.
\begin{lstlisting}[caption={il metodo takeDown di MazziereAgent},label=takedown]
    @Override
    protected void takeDown() {

        say("Felice di aver giocato con voi. Addio!");
        dfd.removeServices(sd);

        if (graphic) {
            gui.dispose();
        }
        if (writeCSV) {
            try {
                csvWriter.close();
            } catch (IOException ex) {
                ex.printStackTrace();
            }
        }
    }
\end{lstlisting}

Gli altri metodo implementati in questa classe sono principalmente metodi di utilità o metodi per interagire con il registro in memoria della partita.
