\section{GeneralAgent}


Questa classe fornisce i metodi di base degli agenti coinvolti nella piattaforma.\\
In concerto con la classe \texttt{GeneralGUI} che implementa i metodi base dell'interfaccia grafica degli agenti, \texttt{GeneralAgent} definisce i metodi per stampare il log delle attività, i messaggi di chat e altre informazioni utili quali l'elenco dei giocatori, dei punteggi ecc.\\
Dispone anche di diversi metodi per l'invio diretto di messaggi. Per una maggiore libertà nelle modalità di invio di messaggi, si è fatto largo ricorso all'\emph{overloading} di questi metodi.\\
Oltre ad altri metodi di varia utilità comuni a tutti gli agenti, \texttt{GeneralAgent} fornisce anche un sistema per l'organizzazione dei \emph{Behaviours}, ovvero dei \emph{task} al livello di framework.\\
Purtroppo \emph{Jade} non fornisce metodi per gestire i \emph{Behaviours} ad alto livello; questo significa che anche solo per ottenere la lista dei \emph{Behaviours} associati ad un agente è necessario agire a livello del sistema operativo.\\
Per evitare questa faticosa e poco elegante soluzione, si è deciso di estendere il metodo \texttt{addBehaviour} \ref{addBehaviour} della classe \texttt{Agent} di Jade nella classe \texttt{GeneralAgent}: in questo modo, aggiungendo un \emph{Behaviour} a partire da un agente mazziere o giocatore, il \emph{Behaviour} viene aggiunto ad una lista all'interno dell'istanza da cui è stato chiamato.

\lstset{numbers=left, numberstyle=\tiny, stepnumber=1,firstnumber=1,
  numbersep=5pt,language=Java,
stringstyle=\ttfamily,
basicstyle=\footnotesize, 
showstringspaces=false,
breaklines=true
}
\begin{lstlisting}[caption={},label=addBehaviour]
protected List<Behaviour> behaviours;
/*    ...      */
    @Override
    public void addBehaviour(Behaviour b) {
        super.addBehaviour(b);
        this.behaviours.add(b);
    }
\end{lstlisting}
