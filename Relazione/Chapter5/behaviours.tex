\section{I \emph{Behaviours}}

Come in ogni sistema multi agente, il cuore dell'interazione fra gli attori del sistema è definito nei \emph{task} assegnati agli agenti.\\
\textsc{Jade} mette a disposizione la classe \texttt{Behaviour} per implementare i \emph{task}, così come alcune sue estensioni per tipi particolari di compiti (maggiori informazioni sono reperibili nell'appendice A).\\
Seguendo questa impostazione, nello sviluppo della nostra piattaforma si sono estese queste classi fornite da \textsc{Jade} per implementare i comportamenti specifici che gli agenti giocatore e mazziere dovessero tenere per interagire in modo da condurre una regolare partita di Briscola in 5.\\
Di seguito indichiamo i principali \emph{Behaviours} implementati per i vari tipi di agente.

Classi usate da entrambi i tipi di agente:
\begin{itemize}
   \item \texttt{GetChatMessage} è un behaviour ciclico che recupera e gestisce i messaggi di chat;
   \item \texttt{GetErrorMessage} come \texttt{GetChatMessage} ma volto ai messaggi di errore;
   \item \texttt{SendAndWait} spedisce messaggi di tipo bloccanti (estende \texttt{SendMessage});
   \item \texttt{SendMessage} spedisce messaggi di tipo semplice.
\end{itemize}

Behaviour specifici per l'agente mazziere:

\begin{itemize}
   \item \texttt{AskBriscola} richiede la briscola al vincitore dell'asta;
   \item \texttt{BeginGame} dà inizio alla partita;
   \item \texttt{DistributeHands} distribuisce le carte;
   \item \texttt{EndGame} gestisce le operazioni da eseguirsi al termine della partita;
   \item \texttt{GetGiocataComment} behaviour ciclico che recupera e gestisce i commenti alle giocate da parte dei giocatori umani;
   \item \texttt{ManageBid} gestisce la fase dell'asta;
   \item \texttt{OfferAChair} risponde alla richiesta di partecipazione da parte di un giocatore non ancora iscritto;
   \item \texttt{OpenTable} si iscrive al servizio pagine gialle e offre i propri servigi;
   \item \texttt{PlayGame} gestisce la fase di gioco;
   \item \texttt{WaitForSubscriptionConfirmation} richiamato dopo \texttt{OfferAChair}, attende limitatamente la conferma d'iscrizione da parte di un giocatore.
\end{itemize}

Behaviour specifici per l'agente giocatore:

\begin{itemize}
   \item \texttt{BeginGame} gestisce la prima fase del gioco;
   \item \texttt{DeclareBriscola} il vincitore dell'asta risponde al mazziere dichiarando il seme di briscola;
   \item \texttt{PlayAuction} gestisce la fase d'asta;
   \item \texttt{PlayGame} gestisce la fase di gioco;
   \item \texttt{ReceiveHand} attende dal mazziere la propria mano di carte;
   \item \texttt{ReceiveScore} attende dal mazziere le informazioni sui punteggi;
   \item \texttt{Subscribe} iscrive il giocatore a un tavolo libero presso un mazziere;
   \item \texttt{WaitForBriscola} il villano attende dal mazziere la comunicazione del seme di briscola.
\end{itemize}


