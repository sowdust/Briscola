
\section{Contributi originali}
I contributi originali presentati in questa tesi sono costituiti principalmente da software.
\subsection{Piattaforma per il gioco della briscola}
Per poter simulare efficacemente una partita di Briscola in 5 per prima cosa si è visto necessario disporre di una piattaforma software che permettesse l'interazione di più agenti di varia natura e fosse in grado di regolare una partita rispettandone i tempi e le regole.\\
Volendo permettere anche a giocatori umani di prendere parte al gioco, si è reso auspicabile lo sviluppo di una piattaforma multiagente che funzioni anche in rete e sia il più possibile indipendente dal sistema sottostante.\\
Per queste motivazioni si è deciso di usare \textsc{Jade - JAVA Agent DEvelopment Framework}, una piattaforma open source per applicazioni di agenti \emph{peer 2 peer} facilmente estendibile con codice \textsc{Java}.
(Maggiori informazioni su \textsc{Jade} sono reperibili nell'appendice A e sul sito ufficiale \cite{jade}).

Il risultato è una piattaforma che mette a disposizione dei tavoli da gioco virtuali con degli agenti "mazziere" che gestiscono le partite, degli agenti virtuali capaci di giocare seguendo diverse strategie facilmente estendibili e un'interfaccia grafica per gli agenti umani.

\subsection{Expert system}

Si sono raccolti e catalogati strategie, informazioni e suggerimenti da conoscenti, manuali, siti internet e forum dedicati alla Briscola in 5.\\
Successivamente si è formalizzata questa conoscenza basilare e tradotta in un sistema a regole scritto in \textsc{Jess -  the Rule Engine for the Java Platform}. (Maggiori informazioni su \textsc{Jess} sono reperibili nell'appendice B e sul sito ufficiale \cite{jess}).\\
Queste regole sono state poi implementate all'interno di un framework sviluppato sempre in \textsc{Jess} che fornisse un ambiente adatto alla loro applicazione.
Tale framework mette a disposizione le regole del gioco insieme a strutture adatta all'analisi della situazione e un sistema che, tramite l'uso di valori di priorità (\emph{salience}) associati a ciascuna regola implementata, fornisca un valido supporto per la \emph{risoluzione dei conflitti} di strategie candidate all'uso, così come la possibilità di variare facilmente priorità ai singoli componenti del ragionamento, in modo da permettere una fase di \emph{"tuning"} dell'insieme di regole.\\
Le regole implementate tentano contemporaneamente di 
\begin{itemize}
  \item analizzare le giocate altrui per farsi un'idea sul ruolo dei singoli giocatori
  \item ad ogni mano scegliere la carta da giocare, tenendo presente quando possibile dei risultati conseguiti nella fase di analisi 
\end{itemize}

\subsection{Interfaccia per l'utente esterno}
In questo lavoro ci si è orientati maggiormente allo sviluppo di un framework che permettesse la raccolta e l'implementazione di strategie piuttosto che alla redazione delle strategie stesse.\\
Per questo motivo nell'interfaccia grafica per i giocatori umani si è introdotta la possibilità di inserire commenti associandoli alle singole giocate effettuate: tali commenti vengono a fine partita salvati insieme al contesto in cui sono stati scritti in un file di log CSV, il quale può successivamente essere analizzato per la traduzione, da parte di un programmatore, dei consigli del giocatore umano in strategie comprensibili ed utilizzabili dall'agente virtuale.
