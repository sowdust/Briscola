\section{Struttura della tesi}

Vengono innanzitutto presentate, nel secondo capitolo, l'insieme delle regole che descrive il gioco \emph{Briscola in 5} nella versione presa in considerazione in questo lavoro.
Nel terzo capitolo si passano brevemente in rassegna alcune fra le più diffuse tecniche di risoluzione applicate ai giochi nell'ambito dell'Intelligenza Artificiale, valutando la loro applicabilità al gioco preso qui in considerazione.
Nel quarto capitolo è descritta l'architettura del sistema multiagente che permette lo svolgimento delle partite.
Il quinto capitolo è dedicato alle strategie usate dall'agente per affrontare il gioco nelle sue diverse parti.
Nel sesto capitolo si dà una breve panoramica sull'implementazione più a "basso livello" della piattaforma multiagente.
Le conclusioni presentano alcune riflessioni sul lavoro fatto così come spunti per possibili sviluppi.
Le appendici forniscono qualche dettaglio sulle tecnologie utilizzate per lo sviluppo della piattaforma multiagente e del reasoner.
