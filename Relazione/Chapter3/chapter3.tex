\chapter{Stato dell'arte}
\graphicspath{{Chapter3/Chapter3Figs/PNG/}{Chapter3/Chapter3Figs/PDF/}{Chapter3/Chapter3Figs/}}

Nell'estate del 1956 si tenne al Dartmouth College (Hanover, New Hampshire, USA) un incontro tra ricercatori americani di teoria degli automi, neti neurali e studi sull'intelligenza, organizzato dagli scienziati John McCarthy, Marvin Minsky, Claude Shannon e Nathaniel Rochester.
Questo incontro è convenzionalmente considerato come la nascita ufficiale degli studi di Intelligenza Artificiale; fra i molti progetti presentati dai partecipanti, ve ne erano alcuni dedicati ai giochi, come ad esempio quello della \emph{dama} \cite{randw}.
Altri lavora ancora precedenti, come il programma che Christopher Strachey scrisse all'Università di Manchester sempre sulla \emph{dama} o quello di Dietrich Prinz per gli scacchi \cite{historyofcomputing}, contribuiscono a dimostrare come l'interesse verso il mondo dei giochi sia stato un filone presente all'interno dell'Intelligenza Artificiale fin dagli albori della disciplina.
Nonostante già i primi programmi per la dama e gli scacchi fossero in grado di tenere testa ad un medio giocatore, negli anni la ricerca ha fatto passi da gigante, culminando con lo sviluppo di \textsc{Deep Blue} da parte dell'\textsc{IBM}, che fu il primo programma informatico a battere un campione mondiale di scacchi (allora il russo \emph{Garry Kasparov}).
Oltre a far aumentare di 18 milione di dollari il capitale azionario dell'\textsc{IBM} (\cite{randw}, l'evento sancì definitivamente la supremazia dell'Intelligenza Artificiale in un certo tipo di giochi.

% ------------------------------------------------------------------------


%%% Local Variables: 
%%% mode: latex
%%% TeX-master: "../thesis"
%%% End: 
