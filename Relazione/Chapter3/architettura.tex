\chapter*{Architettura e interazioni degli agenti}
\stepcounter{chapter}
\addcontentsline{toc}{chapter}{Architettura e interazioni degli agenti}
\graphicspath{{Chapter3/Chapter3Figs/PNG/}{Chapter3/Chapter3Figs/PDF/}{Chapter3/Chapter3Figs/}}

In questo capitolo viene presentata l'architettura generale del sistema e si descrivono le interazioni che i singoli componenti hanno fra di loro.\\
L'architettura è distribuita su diversi agenti indipendenti e progettata in maniera tale da poter funzionare anche in rete.
È costituita da due tipi di agenti: l'agente \textbf{mazziere} e l'agente \textbf{giocatore}.\\
Questi due tipi di agenti comunicano fra di loro direttamente ed esclusivamente tramite scambi di messaggi.\\
La regolamentazione della partita è affidata interamente all'agente \textbf{mazziere}; questa centralizzazione presenta alcuni vantaggi, tra cui:
\begin{itemize}
   \item l'assicurazione che la partita venga svolta secondo le regole, qualsiasi sia il comportamento dei singoli agenti giocatori;
   \item una più semplice gestione di piattaforme in cui si vogliano mantenere aperti dei ``tavoli virtuali'' per un numero imprecisato di giocatori;
   \item l'opportunità, da parte di un agente onniscente, di redigere un file di log di ogni partita, utile per future analisi.
\end{itemize}
\noindent
I singoli agenti possono comunque comunicare tra di loro direttamente tramite un servizio di chat.
