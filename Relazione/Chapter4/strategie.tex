\chapter*{Strategie di gioco}
\stepcounter{chapter}
\addcontentsline{toc}{chapter}{Strategie di gioco}
\graphicspath{{Chapter4/Chapter4Figs/PNG/}{Chapter4/Chapter4Figs/PDF/}{Chapter4/Chapter4Figs/}}

Le due fasi principali del gioco vengono gestite in maniera molto diversa.\\
La \emph{fase dell'asta} viene gestita in maniera totalmente deterministica ed automatica per ogni tipo di giocatore.\\
La \emph{fase di gioco} viene invece gestita in maniere diverse, a seconda del tipo agente giocatore, che può essere \emph{umano}, \emph{casuale} o utilizzare una \emph{strategia da file}.\\
Nel primo caso è il giocatore umano a selezionare di volta in volta, tramite l'interfaccia grafica, la carta da giocare.
Nel caso di un giocatore casuale, invece, viene generato un numero pseudocasuale sulla base del quale viene estratta dal mazzo la carta da giocarsi.\\
Di seguito, sono descritte le strategie usate dal giocatore che implementa il sistema esperto.
Essendo queste ultime soggette a modifiche ed integrazioni, ci si è soffermati particolarmente sul framework che permette l'esecuzione e la codifica delle strategie più che sulle strategie in sè.\\
Si presenta inoltre l'algoritmo usato per gestire la fase d'asta.


