\section{La fase di gioco}


\lstset{numbers=left, numberstyle=\tiny, stepnumber=1,firstnumber=1,
  numbersep=5pt,language=Java,
stringstyle=\ttfamily,
basicstyle=\footnotesize, 
showstringspaces=false,
breaklines=true
}

\begin{lstlisting}
   public void addGiocata(int mano, int counter, Player justPlayer,
                           Card justCard) throws JessException {

        Fact z = new Fact("nuova-giocata", rete);
        z.setSlotValue("player", new Value(justPlayer));
        z.setSlotValue("card", new Value(justCard));
        z.setSlotValue("rank", new Value(justCard.getRank()));
        z.setSlotValue("suit", new Value(justCard.getSuit()));
        rete.assertFact(z);

        if (justCard.equals(briscolaCard)) {
            Fact s = new Fact("socio", rete);
            s.setSlotValue("player", new Value(justPlayer));
            rete.assertFact(s);
            for (Player p : players) {
                if (!p.equals(justPlayer) && !p.equals(this.getPlayer())) {
                    s = new Fact("villano", rete);
                    s.setSlotValue("player", new Value(p));
                    rete.assertFact(s);
                }
            }
        }
        if (justPlayer.equals(this.getPlayer())) {
        
            Fact f = new Fact("in-mano", rete);
            f.setSlotValue("card", new Value(justCard));
            f.setSlotValue("rank", new Value(justCard.getRank()));
            f.setSlotValue("suit", new Value(justCard.getSuit()));
            rete.retract(f);
        } else {
            Fact f = new Fact("in-mazzo", rete);
            f.setSlotValue("card", new Value(justCard));
            f.setSlotValue("rank", new Value(justCard.getRank()));
            f.setSlotValue("suit", new Value(justCard.getSuit()));
            rete.retract(f);
        }

        Fact q = new Fact("in-tavolo", rete);
        q.setSlotValue("card", new Value(justCard));
        q.setSlotValue("player", new Value(justPlayer));
        q.setSlotValue("rank", new Value(justCard.getRank()));
        q.setSlotValue("suit", new Value(justCard.getSuit()));
        rete.assertFact(q);

        rete.run();

\end{lstlisting}



\lstset{numbers=left, numberstyle=\tiny, stepnumber=1,firstnumber=1,
  numbersep=5pt,language=Lisp,
stringstyle=\ttfamily,
basicstyle=\footnotesize, 
showstringspaces=false,
breaklines=true
}


\begin{lstlisting}
   ( defrule nuova-giocata "Ricevo una giocata: aggiorno la situa"
    ?w <- (nuova-giocata (player ?p) (card ?c) (rank ?r) (suit ?s))
    (prende (player ?prende-player) (card ?prende-card))
    ?y <- (giocata-numero ?counter-giocata)
    (seme-mano-fact (suit ?seme-mano))
    (mano-numero ?mano-numero)
=>

   (bind ?tipo "liscio") 

    ;;  Aggiorniamo chi prende
    (if (= ?counter-giocata -1) then
        (remove prende)
        (assert (prende (player ?p) (card ?c) (suit (?c getSuit)) (rank (?c getRank))))
        (remove seme-mano-fact)
        (assert (seme-mano-fact (suit ?s)))
        (bind ?seme-mano ?s)
    else

        (if (batte ?c ?prende-card ?seme-mano )  then
            (remove prende)
            (assert (prende (player ?p) (card ?c) (suit (?c getSuit)) (rank (?c getRank))))
            (if (= ?seme-mano ?s) then
                (if (< (?r getValue) 10) then
                    (bind ?tipo "strozzino")
                else
                    (bind ?tipo "strozzo")
                )
            else
                (bind ?tipo "taglio")
            )
        else
            (if ( > (?r getValue) 9) then
                (bind ?tipo "carico")
            else
                (if (> (?r getValue) 0) then
                    (bind ?tipo "carichino")
                )
            )
        )
    )

    (bind ?new-counter-giocata (+ ?counter-giocata 1))

    (retract ?w)
    (retract ?y)



    ;;  tipi di giocata:
    ;;  -   carichino               J,Q,K   a perdere
    ;;  -   carico                  3,A     a perdere
    ;;  -   strozzo                 3,A     del seme che comanda
    ;;  -   strozzino               carta < 3 del seme che comanda
    ;;  -   taglio                  taglino: quando taglio di briscola
    ;;  -   liscio

    (assert (giocata-numero ?new-counter-giocata))
    (assert (giocata (player ?p) (card ?c) (rank ?r) (suit ?s) (turno ?new-counter-giocata) (mano ?mano-numero) (tipo ?tipo)))
    ;(debug (create$ la giocata di (?c toString) e considerata come ?tipo  ))
)

\end{lstlisting} 
