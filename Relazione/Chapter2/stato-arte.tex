\chapter*{Stato dell'arte}
\stepcounter{chapter}
\addcontentsline{toc}{chapter}{Stato dell'arte}
\graphicspath{{Chapter2/Chapter2Figs/PNG/}{Chapter2/Chapter2Figs/PDF/}{Chapter2/Chapter2Figs/}}

\section{Intelligenza Artificiale e giochi: cenni storici}

Nell'estate del 1956 si tenne al Dartmouth College (Hanover, New Hampshire, USA) un incontro tra ricercatori americani provenienti da campi quali teoria degli automi, reti neurali e studi sull'intelligenza, organizzato dagli scienziati John McCarthy, Marvin Minsky, Claude Shannon e Nathaniel Rochester.
Questo incontro è convenzionalmente considerato come la nascita ufficiale degli studi di Intelligenza Artificiale; fra i molti progetti presentati dai partecipanti, ve ne furono alcuni dedicati ai giochi, come ad esempio quello della \emph{dama}. (\cite{randw})\\
Altri lavori ancora precedenti, come il programma che Christopher Strachey scrisse all'Università di Manchester sempre sulla \emph{dama} o quello di Dietrich Prinz per gli scacchi (\cite{historyofcomputing}), contribuiscono a dimostrare come l'interesse verso il mondo dei giochi sia stato un filone presente all'interno dell'Intelligenza Artificiale fin dagli albori della disciplina.
Nonostante già i primi programmi per la dama e gli scacchi fossero in grado di tenere testa ad un medio giocatore, negli anni la ricerca ha fatto passi da gigante, culminando con lo sviluppo di \textsc{Deep Blue} da parte dell'\textsc{IBM}, che fu il primo programma informatico a battere un campione mondiale di scacchi (il russo \emph{Garry Kasparov} nel 1997).
Oltre a far aumentare di 18 milioni di dollari il capitale azionario dell'\textsc{IBM}, (\cite{randw}) l'evento dimostrò al mondo le potenzialità dell'Intelligenza Artificiale in un certo tipo di giochi nei quali è addirittura in grado di battere i propri creatori.
Grazie alla capacità computazionale del calcolatore su cui gira, infatti, l’algoritmo per il gioco degli scacchi (scritto in C) è capace di calcolare e valutare 100 milioni di posizioni al secondo, una cifra impensabile per un giocatore umano.
È interessante notare come le sue funzioni di valutazione fossero scritte con parametri determinati dal sistema stesso, analizzando migliaia di partite reali di campioni dell'epoca.\\
Con il sofisticarsi degli algoritmi, ma soprattutto con l'aumento della capacità computazionale dei dispositivi, si sono raggiunti risultati straordinari; basti pensare che nel 2008 il software \textsc{Pocket Fritz 3}, installato su uno smartphone, era capace di vincere un torneo di categoria VII, riuscendo a valutare circa 20000 posizioni al secondo. (\cite{pocketfritz})\\
È importante tener presente come il gioco degli scacchi possegga alcune caratteristiche che lo rendono particolarmente adatto ad essere facilmente ed efficientemente risolto per mezzo di alcuni tipi di algoritmi, che verranno presentati nella sezione successiva.\\
Altri tipi di giochi, quali i giochi di carte, non sempre si rivelano così efficientemente trattabili da una macchina.
Questi infatti, a causa dell'incertezza derivante dalle mosse avversarie e dall'aleatorietà con cui le carte coperte sono ordinate nel mazzo e nelle mani altrui, necessitano di approcci differenti.\\
Negli anni il \emph{Bridge} si è affermato come un ottimo caso studio esemplare di alcune delle caratteristiche principali comuni ai giochi di carte.\\
Alcuni dei suoi più efficaci metodi di risoluzione si basano sui già menzionati algoritmi di ricerca abbinati a metodi di decisione basati sui modelli della \emph{teoria dei giochi}. (\cite{frank}, \cite{pavel})
