\chapter*{Stato dell'arte}
\stepcounter{chapter}
\addcontentsline{toc}{chapter}{Stato dell'arte}
\graphicspath{{Chapter2/Chapter2Figs/PNG/}{Chapter2/Chapter2Figs/PDF/}{Chapter2/Chapter2Figs/}}

\section{Intelligenza Artificiale e giochi}

Nell'estate del 1956 si tenne al Dartmouth College (Hanover, New Hampshire, USA) un incontro tra ricercatori americani provenienti da campi quali teoria degli automi, neti neurali e studi sull'intelligenza, organizzato dagli scienziati John McCarthy, Marvin Minsky, Claude Shannon e Nathaniel Rochester.
Questo incontro è convenzionalmente considerato come la nascita ufficiale degli studi di Intelligenza Artificiale; fra i molti progetti presentati dai partecipanti, ve ne furono alcuni dedicati ai giochi, come ad esempio quello della \emph{dama} \cite{randw}.
Altri lavora ancora precedenti, come il programma che Christopher Strachey scrisse all'Università di Manchester sempre sulla \emph{dama} o quello di Dietrich Prinz per gli scacchi \cite{historyofcomputing}, contribuiscono a dimostrare come l'interesse verso il mondo dei giochi sia stato un filone presente all'interno dell'Intelligenza Artificiale fin dagli albori della disciplina.
Nonostante già i primi programmi per la dama e gli scacchi fossero in grado di tenere testa ad un medio giocatore, negli anni la ricerca ha fatto passi da gigante, culminando con lo sviluppo di \textsc{Deep Blue} da parte dell'\textsc{IBM}, che fu il primo programma informatico a battere un campione mondiale di scacchi (il russo \emph{Garry Kasparov} nel 1997).
Oltre a far aumentare di 18 milioni di dollari il capitale azionario dell'\textsc{IBM} (\cite{randw}), l'evento sancì definitivamente la supremazia dell'Intelligenza Artificiale sugli umani in un certo tipo di giochi.
Grazie alla capacità computazionale del calcolatore su cui gira infatti, l’algoritmo per il gioco degli scacchi (scritto in C) è capace di calcolare e valutare 100 milioni di posizioni al secondo, una cifra impossibile per un umano.
È interessante notare come le sue funzioni di valutazione fossero scritte con parametri determinati dal sistema stesso, analizzando migliaia di partite reali di campioni dell'epoca.
Con il sofisticarsi degli algoritmi, ma soprattutto con l'aumento della capacità computazionale dei dispositivi, si sono raggiunti risultati straordinari; basti pensare che nel 2009 il software \textsc{Pocket Fritz 4}, installato su uno smartphone, era capace di vincere un torneo di categoria 6, riuscendo a valutare circa 20000 posizioni al secondo [23].
È importante far presente come il gioco degli scacchi possegga alcune caratteristiche che lo rendono particolarmente adatto ad essere facilmente ed efficientemente risolto per mezzo di alcuni tipi algoritmi che verranno presentati nella sezione successiva.\\




\textbf{FARE ESEMPI DI GIOCHI RISOLTI CON GAME THEORY}\\



Altri tipi di giochi, quali i giochi di carte, non sempre si rivelano così efficientemente trattabili da una macchina.
I giochi di carte, a causa dell'incertezza derivante dalle mosse avversarie e dall'aleatorietà con cui le carte coperte sono ordinate nel mazzo e nelle mani altrui, necessitano di approcci differenti.
Negli anni il \emph{Bridge} si è affermato come caso studio dei giochi di carte nell'Intelligenza Artificiale.\\




\textbf{DEVO SPIEGARE COME IL BRIDGE VIENE RISOLTO?}\\




\section{Ricerca nello spazio degli stati}


È importante notare alcune caratteristiche del gioco degli scacchi che lo rendono particolarmente adatto ad essere risolto efficientemente da un Intelligenza Artificiale; esso infatti è, secondo la definizione di \cite{randw}, un ambiente
\begin{itemize}
   \item Osservabile
   \item Conosciuto
   \item Discreto
   \item Deterministico
\end{itemize}
\emph{Osservabile} si dice di un ambiente le cui caratteristiche salienti (in questo caso le pedine e relative posizioni) sono note all'agente che deve effettuare una scelta.
L'ambiente degli scacchi è \emph{conosciuto} nel senso che le sue "leggi fisiche", ovvero le norme che regolano le possibili mosse delle pedine, sono risapute.
Inoltre è \emph{discreto} perchè può trovarsi in un numero finito di stati distinti gli uni dagli altri; infine è \emph{deterministico} in quanto ogni stato è determinato esclusivamente dallo stato precedente e dall'azione svolta dall'agente.
Inoltre, sempre secondo la definizione di \cite{randw}, gli scacchi sono un ambiente \emph{competitivo}, in quanto un agente (o giocatore), tentando di massimizzare il proprio punteggio, cerca allo stesso tempo di minimizzare quello dell'avversario.
Tralasciando quest'ultima caratteristica, in generale un agente che si trovi a dover raggiungere un \emph{obiettivo} all'interno di un ambiente osservabile, conosciuto, discreto e deterministico, può farlo tramite una \emph{ricerca}.
Per fare ciò è però altresì necessario che il problema sia ben posto.

\subsection{Problemi ben posti}
Un \emph{problema} può essere formalizzato in cinque componenti:
\begin{enumerate}
   \item Uno \emph{stato iniziale} da cui partire.
   \item Una serie di \emph{azioni} o, nel caso degli scacchi, di \emph{mosse}: dato un particolare stato della scacchiera, le mosse che posso fare rispettando le regole del gioco. La funzione virtuale \textsc{ACTIONS(\emph{s})} fornisce l'insieme delle azioni legali dallo a partire stato \emph{s}.
   \item Un \emph{modello di transizione} è una funzione \textsc{RESULT(\emph{s},\emph{s})} che definisce lo stato che si raggiunge applicando l'azione \emph{a} allo stato \emph{s}.
Un \emph{percorso} nello spazio degli stati è una sequenza di stati connesse da una sequenza di azioni
   \item Un \emph{goal test} è una funzione che, applicata ad uno stato, permette di sapere se esso è uno stato obiettivo oppure no.
   \item Una \emph{funzione di costo} è una funzione che assegna un valore numerico (detto \emph{costo}) a ciascun percorso nello spazio degli stati.
\end{enumerate}

\subsubsection*{Esempio applicato al gioco degli scacchi}
Per fare un esempio applichiamo la definizione di \emph{problema ben posto} alla chiusura \emph{Hansen vs. Larsen, Odense 1988}.
L'agente che prendiamo in considerazione è il giocatore nero.
In questo caso l'obiettivo non è quello di dare scacco matto all'avversario (obiettivo impossibile) ma di raggiungere una situazione di pareggio invece che una di sconfitta.

Lo \emph{stato iniziale} \emph{s\textsubscript{0}} è rappresentato dalla situazione in figura \ref{stato-iniziale}.

Le \emph{azioni} \textsc{ACTIONS(\emph{s\textsubscript{0}})}, ovvero le possibili mosse applicabili allo stato iniziale, equivalgono alle sole mosse effettuabili dal re (unica pedina nera in gioco) che, secondo le regole degli scacchi, può spostarsi di una casella in ogni direzione (fig. \ref{azioni}).

Il \emph{modello di transizione} \textsc{RESULT(\emph{s\textsubscript{0}},\emph{a})} è rappresentato in figura \ref{tmodel} per ogni azione legale \emph{a} descritta in \textsc{ACTIONS(\emph{s\textsubscript{0}})}.

Lo stato di \emph{goal} (o \emph{obiettivo}) è rappresentato nella figura \ref{goal}.

\begin{figure}[!htbp]
  \begin{center}
    \leavevmode
      \includegraphics[width=2in]{initial-state}
    \caption{Stato Iniziale \emph{s\textsubscript{0}}}
    \label{stato-iniziale}
  \end{center}
\end{figure}


\begin{figure}[!htbp]
  \begin{center}
    \leavevmode
      \includegraphics[width=2in]{actions}
    \caption{Azioni \textsc{ACTIONS(\emph{s\textsubscript{0}})}}
    \label{azioni}
  \end{center}
\end{figure}



\begin{figure}[!htbp]
  \begin{center}
    \leavevmode
      \includegraphics[width=\textwidth]{transition-model}
    \caption{Stati del modello di transizione raggiungibili da \emph{s\textsubscript{0}}}
    \label{tmodel}
  \end{center}
\end{figure}


\begin{figure}[!htbp]
  \begin{center}
    \leavevmode
      \includegraphics[width=2in]{goal}
    \caption{Stato \emph{goal}}
    \label{goal}
  \end{center}
\end{figure}



\subsection{Ricerca della soluzione}

Dato un problema ben posto è quindi possibile rappresentarne lo \emph{spazio degli stati} come un albero in cui
\begin{itemize}
   \item la \emph{radice} rappresenta lo \emph{stato iniziale}
   \item i \emph{rami} sono le \emph{azioni applicabili}
   \item i \emph{nodi} sono gli stati appartenenti allo spazio degli stati del problema
\end{itemize} 


Per costruire tale albero è possibile procedere per successiva \emph{espansione dei nodi}: dato lo stato rappresentato da un nodo, ad essi vengono applicate tutte le azioni possibili e, tramite il modello di transizione, si ricavano gli stati successivi che diventeranno ulteriori nodi figli del precedente.
L'insieme dei nodi raggiunti ma ancora da espandere viene chiamato \emph{frontiera}.
Viene detto \emph{branching factor} il numero (esatto o medio) di figli di ogni nodo, ovvero di azioni applicabili ad uno stato.
È immediato notare come un branching factor elevato renda praticamente impossibile la costruzione ed esplorazione dell'intero spazio degli stati, data la natura esponenziale della crescita dei nodi da un livello al successivo.
Questo vale ovviamente anche per il gioco degli scacchi, il cui branching factor - ovvero il numero medio di mosse che un giocatore può effettuare - è stato calcolato essere 35 (\cite{chessbf}).
Per questo motivo è necessario che gli algoritmi di ricerca implementino delle strategie utili a espandere solo i nodi considerati più convenienti per raggiungere un nodo goal.


\section{Teoria dei giochi}


% ------------------------------------------------------------------------


%%% Local Variables: 
%%% mode: latex
%%% TeX-master: "../thesis"
%%% End: 
