\section{Il caso particolare della briscola in 5}

La briscola in 5 presenta una peculiarità rispetto agli altri giochi: la suddivisione dei cinque giocatori in due squadre, necessariamente assimetriche, non avviene prima dell'inizio della partita, bensì all'interno della stessa.
Inoltre, i giocatori posseggono una diversa quantità d'informazione riguardante la formazione delle squadre, che può rimanere incerta per gran parte della partita.
Per questo è auspicabile considerare le due fasi del gioco separatamente: dapprima vi è una fase in cui la formazione delle squadre non è nota; una volta che il \emph{socio} gioca la carta chiamata, palesandosi, ha inizio la seconda fase, durante la quale i ruoli di tutti i giocatori diventano chiari.
La seconda fase del gioco quindi non si discosta troppo da altri giochi già molto trattati nell'ambito dell'AI, come il \emph{Bridge}.
Uno studio di \cite{villa} ad esempio, applica questa fase della briscola in 5 il metodo della ricerca nello spazio degli stati con approssimazione Montecarlo, soluzione già indicata come molto comune nella risoluzione di giochi di carte.
Non siamo invece riusciti a trovare studi che trattino la prima fase della briscola in 5, quella durante la quale non è chiaro a tutti chi siano gli avversari e chi i compagni.
\textbf{NOTA ---- È DAVVERO ZERO-SUM? I PUNTEGGI SOMMANO A 0 E SONO DI SEGNO OPPOSTO MA NON UGUALI!}
Pur essendo uno \emph{zero-sum game}, infatti, la briscola in 5 nella sua prima fase non può essere considerato un ambiente nè \emph{competitivo} nè \emph{cooperativo}.
Questo rende impossibile l'applicazione dei metodi indicati nelle precedenti sezioni e chiama un approccio di tipo differente.

\subsection{Inapplicabilità del minimax}
Volendo tentare di applicare l'algoritmo \textsc{Minimax} alla risoluzione di una partita di briscola in 5 ci si scontra immediatamente con una difficoltà insormontabile: essendo impossibile definire con certezza una funzione di utilità, è inapplicabile la definizione ricorsiva \ref{minimax-value} del \emph{mini-max value}.
Questo accade perchè, pur assumendo di aver costruito l'intero albero rappresentante lo spazio degli stati fino alle foglie che contengono gli stati finali della partita, è impossibile assegnare un segno, positivo o negativo, al modulo delle funzioni di utilità: esso sarebbe positivo se il giocatore che si aggiudica la presa fosse un compagno, viceversa, sarebbe negativo.
Ma essendo sconosciuto il ruolo del giocatore che prende, rimane incerto il valore della funzione di utilità.\\
Per poter applicare l'approccio \textsc{Minimax} anche in questo scenario, si renderebbe necessario aumentare il numero di mondi possibili, moltiplicandolo per 3 (il numero dei giocatori dei quali non si conosce il ruolo) per considerare tutte le situazioni possibili.
Ma in questo caso, essendo le funzioni di utilità dei differenti mondi diametralmente opposte, sarebbe anche auspicabile possedere delle valide euristiche che permettano di assegnare con buon grado di confidenza una probabilità ad ogni possibile "mondo terminale" (ovvero alla possibilità che ognuno dei 3 giocatori di cui non si conosce il ruolo sia il socio).

\subsubsection*{Esempio}
Immaginiamo una situazione in cui un giocatore, $P$, nel ruolo di \emph{villano}, durante la fase in cui il \emph{socio} non si sia ancora rivelato, si trovi ultimo di mano in una partita contro i giocatori $A$, $B$, $C$ e $G$. $G$ sia il \emph{chiamante}.\\

In tavolo non vi siano briscole; il seme regnante sia $\heartsuit$.\\
La presa spetti al giocatore, $A$ del quale ancora non si conosce il ruolo.\\
$A$ prenda con un Asso di $\heartsuit$.\\
In tavolo vi siano anche: K $\heartsuit$, J $\diamondsuit$, 2 $\spadesuit$.\\
Il seme di briscola sia $\clubsuit$.\\
In mano si abbiano 2 $\clubsuit$ e 2 $\heartsuit$ : una briscola ed un carico.\\
Immaginiamo, per facilitare i calcoli, che tutti i punti siano già stati giocati e che rimanga una sola mano da giocarsi.\\
In una situazione del genere un giocatore umano incerto sui ruoli degli altri giocatori giocherebbe certamente la briscola (quindi il 2$\clubsuit$), aggiudicandosi la mano senza per questo dover rinunciare a dei punti.\\
Ma la situazione è utile per mostrare come l'inapplicabilità della definizione del \emph{minimax value}: se si decidesse di giocare il 2 $\heartsuit$, infatti, i punti raccolti dal giocatore $A$ che ha giocato A $\heartsuit$, sarebbero a proprio vantaggio o svantaggio?\\
Giocando il 2 $\clubsuit$ si è certi che, qualsiasi sia lo svolgersi della mano successiva, il punteggio della propria squadra aumenterà di 16 punti.\\
Giocando il 2 $\heartsuit$, invece, i 16 punti della mano potrebbero essere a favore della propria squadra come di quella avversaria, in base all'affiliazione del giocatore che si aggiudica la presa.\\


\Tree[.{$P$ ha in mano \\ $2  \heartsuit$ e $ 2  \clubsuit$}
         [.{$P$ gioca $2 \heartsuit$} [ .{...} [.{termine partita} [ .{Il socio è A} [ {- 16} ] ] [ .{Il socio è B}  [ {+ 16} ]] [ .{Il socio è C}  [ {+ 16} ]] ]]]
         [.{$P$ gioca $2 \clubsuit$} [ .{...} [.{termine partita} [ .{Il socio è A}  [ {+ 16} ]] [ .{Il socio è B}  [ {+ 16} ]] [ .{Il socio è C}  [ {+ 16} ]] ]]]
%           [.{$2 \clubsuit$} [.I \textsc{3sg.Pres} ]
%                 [.VP [.V\1 [.V \textit{is} ]
%                            [.AP [.Deg \textit{really} ]
%                                 [.A\1 [.A \textit{simple} ]
%                                       \qroof{\textit{to use}}.CP ]]]]]
]



