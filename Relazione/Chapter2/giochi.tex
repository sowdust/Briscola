\section{Teoria dei giochi}

La \emph{Teoria dei giochi} è una branca della matematica che studia "modelli di conflitto e cooperazione tra agenti razionali capaci di intraprendere decisioni" (tradotto da \cite{gtheory}), ovvero le strategie da seguire per effettuare decisioni in maniera ottimale.\\
Gli ambiti applicativi delle teoria dei giochi sono svariati: Economia, Politica, Logica, Biologia, Informatica e persino Psicologia.\\
Sviluppatasi nella prima metà del Novecento, questa disciplina conobbe un forte aumento di interesse da parte dei ricercatori a partire dagli anni '50, dopo la pubblicazione del volume "Theory of Games and Economic Behavior" da parte del matematico John von Neumann insieme all'economista Oskar Morgenstern \cite{tog}. Agli inizi degli anni 2000 la Teoria dei giochi ebbe modo di essere conosciuta anche dal pubblico non specialista grazie alla celebre pellicola holliwoodiana \emph{A Beautiful Mind}, ispirata alla vita e agli studi del famoso matematico John Nash, vincitore del premio Nobel per l'Economia nel 1994 con lavori di Teoria dei giochi (\cite{jnash}).


\subsection{Rappresentazione dei giochi}

Un gioco può essere formalmente definito da quattro elementi (\cite{ramusen}):
\begin{enumerate}
   \item i \emph{giocatori} o, più in generale, gli agenti partecipanti al processo;
   \item le \emph{informazioni} disponibili ai giocatori in ciascun momento decisivo;
   \item le \emph{azioni} a disposizione dei giocatori in ciascun momento decisivo;
   \item i \emph{payoff} o le funzioni di utilità per ogni giocatore, per ogni possibile risultato finale.   
\end{enumerate}

A partire da questa formalizzazione, lo studioso applica al gioco una regola formale che descrive le sue previsioni su come il gioco sarà svolto dai vari giocatori, ovvero quali strategie verranno adottate; tali regole vengono dette \emph{solution concept} (\cite{wikisol}).\\
Tramite questo processo si cercano di dedurre delle strategie di \emph{equilibrio} tali che, se adottate, nessun giocatore può guadagnare punti rispetto agli altri deviando da esse.


\subsubsection*{Esempio applicato}


\subsection{Metodo Risolutivo}



\subsection{Nei giochi di carte}



