\section{Teoria dei giochi}

La \emph{Teoria dei giochi} è una branca della matematica che studia ``modelli di conflitto e cooperazione tra agenti razionali capaci di ``intraprendere decisioni'', \cite{gtheory} ovvero le strategie da seguire per effettuare decisioni in maniera ottimale.\\
Gli ambiti applicativi delle teoria dei giochi sono diversi: Economia, Politica, Logica, Biologia, Informatica e persino Psicologia.\\
Sviluppatasi nella prima metà del Novecento, questa disciplina conobbe un forte aumento di interesse da parte dei ricercatori a partire dagli anni '50, dopo la pubblicazione del volume ``Theory of Games and Economic Behavior'' da parte del matematico John von Neumann insieme all'economista Oskar Morgenstern. % (\cite{tog}).
Agli inizi degli anni 2000 la Teoria dei giochi ebbe modo di essere conosciuta anche dal pubblico non specialista grazie alla celebre pellicola holliwoodiana \emph{A Beautiful Mind}, ispirata alla vita e agli studi del famoso matematico John Nash, vincitore del premio Nobel per l'Economia nel 1994 con lavori di Teoria dei giochi. (\cite{jnash})


\subsection{Rappresentazione dei giochi}

Un gioco può essere formalmente definito da quattro elementi: (\cite{ramusen})
\begin{enumerate}
   \item i \emph{giocatori} o, più in generale, gli agenti partecipanti al processo;
   \item le \emph{informazioni} disponibili ai giocatori in ciascun momento decisivo;
   \item le \emph{azioni} a disposizione dei giocatori in ciascun momento decisivo;
   \item i \emph{payoff} o le funzioni di utilità per ogni giocatore, per ogni possibile risultato finale.   
\end{enumerate}
A partire da questa formalizzazione, si  può applicare al gioco una regola formale che descrive le sue previsioni su come il gioco sarà svolto dai vari giocatori, ovvero quali strategie verranno adottate; tali regole vengono dette \emph{solution concept}. Tramite questo processo si cercano di dedurre delle strategie che conducano a situazioni di \emph{equilibrio}.



\subsection{Forma normale e forma estesa}
I giochi in cui i partecipanti agiscono contemporaneamente, o comunque senza conoscere le azioni degli avversari, sono detti \emph{simultanei}.
Il modello di tali giochi viene generalmente rappresentato in \textbf{forma normale}, cioè come una matrice che rappresenti i giocatori, le possibili strategie e i relativi \emph{payoff} (un esempio: \ref{prigionierofn}).\\
Se il gioco è invece di tipo \emph{sequenziale}, allora si usa la \textbf{forma estesa}: un albero di decisione finito che ha per nodi le possibili mosse (vedi \ref{extensive}).
Le mosse possono essere \emph{personali} o \emph{casuali}: le prime sono le mosse direttamente riconducibili alla scelta di un giocatore, le seconde invece a un evento ambientale ai cui possibili esiti è associata una precisa probabilità; fanno parte di questo genere di mosse il lancio di un dado o l'estrazione di una carta da un mazzo non ordinato.
Un nodo posto alla radice, generalmente rappresentato con la forma di un diamante, rappresenta lo stato iniziale $S_0$ del gioco.
Un ``gioco'' (o \emph{play}) $\alpha$ viene realizzato a partire dal nodo radice, effettuando una scelta (da parte dei giocatori o del caso) ad ogni ramo e terminando in un nodo foglia.\\
In questo caso il \emph{payoff} può essere assegnato tramite una \emph{funzione di utilità} del gioco, così che il \emph{payoff} del giocatore $i$ sul gioco $\alpha$ secondo la utility function $K$ è dato dal valore $K_i(\alpha)$. \cite{frank}

\subsection{Il concetto di equilibrio}

\begin{quote}
Un gioco può essere descritto in termini di strategie che i giocatori devono seguire nelle loro mosse: l'equilibrio c'è, quando nessuno riesce a migliorare in maniera unilaterale il proprio comportamento. Per cambiare, occorre agire insieme. \emph{John Nash}
\end{quote}
La precedente citazione introduce in maniera informale il concetto di \emph{equilibrio}, cardine della teoria dei giochi, secondo il quale la migliore strategia è quella che consente il miglior \emph{payoff} a tutti i partecipanti senza offrire a nessuno di essi la possibilità di deviare da tale strategia guadagnandoci.\vspace*{.7cm} \\
Più formalmente si può definire (\cite{eponimo}) il concetto come segue, caratterizzando un gioco come:
\begin{itemize}
   \item un insieme $G$ di giocatori o agenti, di cardinalità $N$ 
   \item un insieme $S$ di strategie, costituito a sua volta da un insieme di $N$ vettori $S_i$, ciascuno dei quali contenga l'insieme delle strategie che il giocatore $i$-esimo ha a disposizione; si indichi con $s_i$ la strategia scelta dal giocatore $i$;
\begin{center}
   $S_i = (s_{i,1},s_{i,2},...s_{i,j},...s_{i,M_i})$
\end{center}   
\item Un insieme $U$ di funzioni
\begin{center}
   $u_i=U_i\left(s_1, s_2,...,s_i,...,s_N\right)$
\end{center}
che associno ad ogni giocatore $i$ il payoff $u_i$ derivante da una data combinazione di strategie (proprie e degli avversari).
\end{itemize}


Un equilibrio di Nash per un gioco così definito è una combinazione di strategie
\begin{center}
   $s_1^*, s_2^*,...,s_N^*$
\end{center}

tale che

\begin{center}
    $  \forall i \forall s_i :   U_i\left(s_1^*, s_2^*,...,s_i^*,...,s_N^*\right)\ge U_i\left(s_1^*, s_2^*,...,s_i,...,s_N^*\right)$.
\end{center}

%per ogni $i$ e per ogni strategia $s_i$ scelta dal giocatore $i$-esimo.

\noindent
Informalmente: se un gioco ammette un equilibrio di Nash, ogni agente ha a disposizione una strategia $s^*$ che massimizza il proprio payoff quando anche tutti gli altri agenti stiano seguendo una strategia scelta con lo stesso criterio.\\
Da questa definizione è facile comprendere il significato intuitivo di strategia di equilibrio come strategia che non permetta agli altri partecipanti di deviare da essa senza diminuire il proprio payoff: l'unica variabile su cui il giocatore $i$ ha influenza nella formula finale è la scelta della propria strategia; ma $s_i^*$ costituisce già la strategia che consente di massimizzare il proprio payoff.


\subsubsection*{Esempio: il dilemma del prigioniero}

Un famoso esempio di situazione in cui è utile l'applicazione della teoria dei giochi è il \emph{Prisoner's Dilemma}, formulato per la prima volta da Merrill Flood e Melvin Dresher nel 1950 e in seguito formalizzato nella versione qui presentata da Albert W. Tucker nel 1992.
Il dilemma è il seguente:

\begin{quote}
   Due criminali appartenenti alla stessa banda vengono arrestati e imprigionati. Entrambi si trovano in una situazione di isolamento senza alcuna possibilità di scambiare messaggi con l'altro. Gli accusatori non hanno le prove necessarie per condannare i due in base all'accusa principale; i due infatti sperano di essere condannati a un anno di prigione per l'accusa minore.\\
   Gli accusatori però offrono loro l'opportunità di tradire ciascuno l'altro per vedersi ridotta la pena.
   Queste le condizioni precise:
   \begin{itemize}
      \item se entrambi tradiscono, vengono condannati a scontare 2 anni di pena;
      \item se uno solo dei due tradisce l'altro, il traditore viene immediatamente liberato, mentre l'altro condannato a 3 anni;
      \item se nessuno dei due accusa l'altro, entrambi vengono condannati a un anno di pena.
   \end{itemize}
   
\end{quote}



\begin {table}
\begin{center}
  \begin{tabular*}{1\textwidth}{@{\extracolsep{\fill}} | l || c | c | }
    \hline
                     &  B non tradisce    &  B tradisce     \\ \hline
    A non tradisce   & A: 1; B: 1         &  A: 3; B: 0     \\ \hline
    A tradisce       & A: 0; B: 3         &  A: 2; B: 2     \\ \hline 
  \end{tabular*}
  \caption {Dilemma del prigioniero in forma normale; per i prigionieri A e B sono indicati come \emph{payoff} (in questo caso negativi) gli anni di carcere cui vanno incontro nelle varie situazioni possibili} \label{prigionierofn} 
\end{center}
\end {table}
\noindent
Si chiamino i due prigionieri $A$ e $B$.
La situazione è descritta dalla forma normale \ref{prigionierofn} e dalla forma estesa \ref{extensive}.
Si può notare che entrambi i prigionieri ottengono un maggior \emph{payoff} tradendo il compagno piuttosto che cooperando con lui rimanendogli fedele.
Si consideri il prigioniero $B$: può tradire il compagno o cooperare con lui.
Se $A$ collaborasse, a $B$ converrebbe tradire, dato che in questo modo uscirebbe di prigione invece che scontare un anno di pena.
Se invece $A$ tradisse, anche a $B$ converrebbe farlo, dato che sconterebbe un solo anno invece che 3.
Si conclude che in ogni caso al giocatore $B$ conviene tradire.
Lo stesso ragionamento può essere seguito simmetricamente per $A$.


\begin{center}
   
\begin{figure}[!htbp]



\begin{tikzpicture}[every tree node/.style={draw,circle},
   level distance=1.25cm,sibling distance=.5cm, 
   edge from parent path={(\tikzparentnode) -- (\tikzchildnode)}]
\tikzset{level 1/.style={level distance=75pt}}
\tikzset{level 2+/.style={level distance=50pt}}
\Tree [.\node(Root)[diamond] {stato iniziale};
   [.\node(Root2) {0};  
   \edge node[auto=right] {$A$ tradisce}; 
    [.\node (one) {1};
      \edge node[auto=right] {$B$ tradisce};  
      [.\node[rectangle,red] {$A$: 2; $B$: 2};  ]
      \edge node[auto=left] {$B$ non tradisce};  
      [.\node[rectangle] {$A$: 0; $B$: 3}; ] 
    ]
    \edge node[auto=left] {$A$ non tradisce}; 
    [.\node (two) {2};
      \edge node[auto=right] {$B$ tradisce};  
      [.\node[rectangle] {$A$: 3; $B$: 0};  ]
      \edge node[auto=left] {$B$ non tradisce};  
      [.\node[rectangle] (leaf) {$A$: 1; $B$: 1}; ] 
    ] 
]]

   \begin{scope}[xshift=3in,every tree node/.style={},edge from parent path={}]
\Tree [.{ } [.{$Sceglie$ A} [.{$Sceglie$ B} ]]]
\end{scope}

\end{tikzpicture}

    \caption{Il dilemma dei prigionieri in forma estesa; in rosso la situazione di equilibrio}
    \label{extensive}
\end{figure}
\end{center}


\subsection{Game theory per i giochi di carte}

Secondo la classificazione della teoria dei giochi, i giochi di carte appartengono alle seguenti categorie di giochi:

\begin{itemize}
   \item \emph{asimmetrici}: i payoff risultanti da una stessa strategia dipendono da quale giocatore la segue;
   \item \emph{sequenziali} in quanto si sviluppano nel tempo e i giocatori hanno la possibilità di muovere uno alla volta dopo aver visto le mosse altrui;
   \item \emph{a informazione imperfetta}  perchè le carte nelle mani altrui non sono visibili.
\end{itemize}
\noindent
Nonostante alcuni giochi di carte siano sia \emph{competitivi} che \emph{cooperativi}, come ad esempio il \emph{Bridge} o la \emph{Briscola a 4} in cui due squadre di giocatori fra loro cooperanti si affrontano competitivamente, è generalmente possibile semplificare le dinamiche del gioco considerando i giocatori di una stessa squadra come un unico agente, riconducendo l'ambiente ad uno semplicemente \emph{competitivo}. (\cite{pavel})\\
Per la classificazione cui appartengono, i giochi di carte sono adatti ad essere rappresentati in forma estesa, tramite un \emph{albero di decisione} che ne rappresenti tutte le possibili mosse da parte di tutti i giocatori e comprenda per i nodi foglia, cioè le mosse finali, i relativi \emph{payoff}.\\
Essendo i giochi di carte ad \emph{informazione imperfetta} sorgono alcune complicazioni nella costruzione della forma estesa del gioco.
Infatti, alcune scelte possono non essere direttamente disponibili al giocatore; questo può coinvolgere sia le scelte effettuate dal caso, come quando si tratti della distribuzione delle carte in mano all'avversario, sia le scelte dell'avversario stesso, come quando questo giochi una carta mantenendola però coperta.\\
Questo fa sì che un giocatore possa non sapere dove si trovi all'interno dell'albero della forma estesa.\\
Nonostante alcune soluzioni per superare queste difficoltà siano già state formulate, (si veda \cite{wiley}) per la maggior parte dei giochi di carte l'albero rimarrebbe comunque troppo esteso per poter essere trattato computazionalmente in pratica. (\cite{frank})\\
Anche in questo caso quindi, valgono le considerazioni fatte per gli algoritmi di ricerca e la possibilità di utilizzare strategemmi approssimativi unite ad algoritmi di pruning (\cite{pavel}) per rendere l'esplorazione eseguibile in un tempo ragionevole.
