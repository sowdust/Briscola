\chapter*{Appendice B: \textsc{Jess}}
\stepcounter{chapter}
\addcontentsline{toc}{chapter}{Appendice B: Jess}
\textsc{Jess} è insieme un \emph{rule egine} (\emph{motore inferenziale}) e un ambiente di scripting, scritti interamente in \textsc{Oracle Java} da Ernest Friedman-Hill per i Sandia National Laboratories di Livermore, California.\\
Con \textsc{Jess} è possibile creare del software \textsc{Java} che sia capace di ``ragionare'' secondo regole scritte in maniera dichiarativa.\\
Pur essendo un sistema leggero, \textsc{Jess} risulta essere uno dei più veloci motori referenziali in circolazione. Ulteriore potenzialità è l'accesso completo all'insieme di API \textsc{Java} a partire dal linguaggio di scripting.\\
Per la selezione delle regole da attivarsi, il motore inferenziale utilizza una versione modificata dell'algoritmo \textsc{Rete}.
Quest'ultimo è un meccanismo molto efficiente che permette di risolvere il complesso problema del matching molti-a-molti  (si veda, a riguardo, \cite{rete}).\\
Alcuni punti di forza di \textsc{Jess} rispetto ad altri sistemi dello stesso tipo includono la possibilità di effettuare il \emph{backward chaining} delle regole, di eseguire query dirette alla memoria di lavoro e di ragionare e manipolare direttamente oggetti \textsc{Java}.
È disponibile liberamente per uso accademico.
